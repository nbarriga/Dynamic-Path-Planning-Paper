\chapter{Conclusions and Future Work}
\label{sec:conclusions}

The new multi-stage algorithm proposed here has good performance in
very dynamic environments. It behaves particularly well when several small
obstacles are moving around at random. This is explained by the fact
that if the obstacles are constantly moving, they will sometimes move out of the
way by themselves, which our algorithm takes advantage of, while RRT based
ones do not, they just drop branches of the tree that could prove useful
again just a few moments later. The combined \mbox{RRT-EP/N}, although having more
operators, and automatic adjustment of the operator probabilities according to
their effectiveness, is still better than the RRT variants, but about 55\% 
slower than the
simple multi-stage algorithm. This is explained by the number of collision
checks performed, more than twice than the multi-stage algorithm, because
collision checks must be performed for the entire population, not just a single
path.

In the partially known environment, even though the difference in collision checks
is even greater than in dynamic environments, the RRT-EP/N performance is
about 25\% worse than the multi-stage algorithm. Overall, the RRT variants
are closer to the performance of both combined algorithms.

In the totally unknown environment, the combined RRT-EP/N is about 30\% faster
than the simple multi-stage algorithm, and both outperform the RRT variants,
with much smaller times and standard deviations.

All things considered, the simple multi-stage algorithm is the best choice in
most situations, with
faster and more predictable planning times, a higher success
rate, fewer collision checks performed and, above all, a much simpler
implementation than all the other algorithms compared.

This thesis shows that a multi-stage approach, using different techniques for
initial plannning and navigation, outperforms current probabilistic sampling
techniques in dynamic, partially known and unknown environments.

Part of the results presented in this thesis are published in \cite{Barriga09}.

\section{Future Work}
We propose several areas of improvement for the work presented in this thesis.
\subsection{Algorithms}
The most promising area of improvement seems to be to experiment with different on-line planners
such as a version of the
EvP (\cite{Alfaro05} and~\cite{Alfaro08}) modified to work in
continuous configuration space or a potential field navigator. Also, the local
search presented here could benefit from the use of more sophisticated
operators and the parameters for the RRT variants (such as forest size for
MP-RRT),
and the EP/N (such as population size) could benefit from being tuned specifically for this
implementation, and not simply reusing the parameters found in previous work.

Another area of research that could be tackled is extending this algorithm to
higher dimensional problems, as RRT variants
are known to work well in higher dimensions.

Finally, as RRT variants are suitable for kinodynamic planning, we only need to adapt
the on-line stage of the algorithm to have a new multi-stage planner for problems
with kinodynamic constraints.
\subsection{Framework}
The MoPa framework could benefit from the integration of a third party logic
layer, with support for arbitrary geometrical shapes, a spatial scene graph and
hierarchical maps. Some candidates would be
OgreODE~\cite{OgreODE}, Spring
RTS~\cite{SpringRTS} and
ORTS~\cite{ORTS}.

Other possible improvements are adding support for other map formats, including
discrimination of static and moving obstacles, limited
sensor range simulation and integration with external hardware such as the Lego
NXT~\cite{LegoNXT}, to run experiments in a more
realistic scenario.
