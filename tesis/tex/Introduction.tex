\chapter{Introduction}
The \emph{dynamic path-planning} problem consists in finding a suitable
plan for each new configuration of the environment by recomputing a
collision-free path using the new information available at each time step~\cite{Hwang92}.
This kind of problem has to be solved for example by a robot trying to navigate
through an area crowded with people, such as a shopping mall or supermarket.
The problem has been widely addressed in its several flavors,
such as cellular decomposition of the configuration space~\cite{Stentz95},
partial environmental knowledge~\cite{Stentz94},
high-dimensional configuration spaces~\cite{Kavraki96}
or planning with non-holonomic constraints~\cite{Lavalle99}.
However, even simpler variations of this problem are complex enough
that they can not be solved
with deterministic techniques, and therefore are worthy of study.

This thesis is focused on algorithms for
finding and traversing a collision-free path in
two dimensional space, for a
holonomic robot\footnote{A holonomic robot is a robot in
which the controllable degrees of freedom is equal to the total degrees of
freedom.},
without kinodynamic restrictions\footnote{Kinodynamic planning is a problem in which velocity and acceleration
bounds must be satisfied}, in a highly dynamic environment, but for comparison
purposes three different scenarios will be tested:
\begin{itemize}
\item Several unpredictably moving obstacles or adversaries.
\item Partially known environment, where some obstacles become visible when the
robot approaches each one of them.
\item Totally unknown environment, where every obstacle is initially invisible
to the planner, and only becomes visible when the robot approaches it.
\end{itemize}
Besides the obstacles in the second and third scenario we assume
that we have perfect information of the environment at all times.

We will focus on continuous space algorithms and
will not consider algorithms that use a discretized representation of the
configuration space\footnote{the space of possible positions that a physical
system may attain}, such
as D*~\cite{Stentz95},
because for high dimensional problems the
configuration space becomes intractable in terms of both memory and computation
time, and there is the extra difficulty of calculating the discretization size,
trading off accuracy versus computational cost.
Only single agent algorithms will be considered here.
On-line as well as off-line algorithms will be studied. An on-line algorithm is
one that is permanently adjusting its solution as the environment changes, while
an off-line algorithm computes a solution only once (however, it can be executed 
many times).

The offline Rapidly-Exploring Random Tree (RRT) is efficient at finding solutions, but the results are far from
optimal, and must be post-processed for shortening, smoothing or other qualities
that might be desirable in each particular problem. Furthermore, replanning RRTs
are costly in terms of computation time, as are evolutionary and cell-decomposition
approaches. Therefore, the novelty of this work is
the mixture of the feasibility benefits of the RRTs, the repairing capabilities
of local search, and the computational inexpensiveness of greedy algorithms,
into our lightweight multi-stage algorithm. Our working hypothesis will be that
a multi-stage algorithm, using different techniques for initial planning and
navigation,
outperforms current probabilistic sampling techniques in highly dynamic
environments

\section{Problem Formulation}

At each time-step, the problem could be defined as
an optimization problem with satisfiability constraints.
Therefore, given a path our objective is to minimize an evaluation function
(i.e., distance, time, or path-points), with the \(C_{\text{free}}\) constraint.
Formally, let the path \mbox{$\rho=p_1p_2\ldots p_n$} be a sequence of points, where
\mbox{$p_i \in \mathbb{R}^n$} is a $n$-dimensional point (\mbox{\(p_1 = q_{\text{init}}, p_n =
q_{\text{goal}}\)}),
$O_t\in \mathcal{O} $ the set of obstacles positions
at time $t$, and \mbox{$\operatorname{eval} \colon \mathbb{R}^n \times
\mathcal{O} \mapsto \mathbb{R}$}
an evaluation function of the path depending on the object positions.
Our ideal objective is to obtain the optimum $\rho^*$ path that
minimizes our $\operatorname{eval}$ function within a feasibility restriction in the form
\begin{equation}
\displaystyle\rho^*=\underset{\rho}{\operatorname{argmin}}[\operatorname{eval}(\rho,O_t)]  \textrm{ with }
\operatorname{feas}(\rho,O_t) = C_{\text{free}}
\label{eq:problem}
\end{equation}
where \(\operatorname{feas}(\cdot,\cdot)\) is a \emph{feasibility} function that equals
\(C_{\text{free}}\)
if the path $\rho$ is collision free for the obstacles $O_t$.
For simplicity, we use very naive \(\operatorname{eval}(\cdot,\cdot)\) and
\(\operatorname{feas}(\cdot,\cdot)\)
functions, but our approach could be extended easily to more complex evaluation and
feasibility functions.
The \(\operatorname{feas}(\rho,O_t)\) function used assumes that the robot is a point
object in space, and therefore if no segments
\(\overrightarrow{p_i p_{i+1}}\)
of the path collide with any object \(o_j \in O_t\), we say that the path
is in \(C_{\text{free}}\).
The \(\operatorname{eval}(\rho,O_t)\) function is the length of the path, i.e., the sum of the distances between
consecutive points. This could be easily changed to any other metric such as the time
it would take to traverse this path, accounting for smoothness,
clearness or several other optimization criteria.

\section{Document Structure}

In the following sections we present several path planning methods that can be
applied to the problem described above. In section~\ref{sec:RRT} we review
the basic offline, single-query RRT, a probabilistic method that builds a
tree along the free configuration space until it reaches the goal state.
Afterwards, we introduce the most popular replanning variants of
RRT: Execution Extended RRT (ERRT) in section~\ref{sec:ERRT}, Dynamic RRT (DRRT) in
section~\ref{sec:DRRT} and Multipartite RRT (MP-RRT) in section~\ref{sec:MPRRT}.
The Evolutionary Planner/Navigator (EP/N), along with some variants, is presented
in section~\ref{sec:EP/N}.
Then, in section~\ref{sec:RRT-EP/N} we present a mixed approach, using a RRT to
find an initial solution and the EP/N to navigate, and finally, in section~\ref{sec:RRT-LP} we
present our new hybrid multi-stage algorithm, that uses RRT for initial
planning and informed local search for navigation, plus a simple greedy
heuristic for optimization. Experimental results and
comparisons that show that this combination of simple techniques provides better
responses to highly dynamic environments than the standard RRT extensions are
presented in section~\ref{sec:results}. The
conclusions and further work are discussed in section~\ref{sec:conclusions}.
