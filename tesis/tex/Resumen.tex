\prefacesection{Resumen}
Este documento es una tesis en el tema de planificaci\'on de caminos uniagente y en
l\'inea, para ambientes continuos, impredecibles y altamente din\'amicos. El problema es encontrar
y recorrer un camino sin colisiones para un robot holon\'omico, sin restricciones kinodin\'amicas,
movi\'endose en un ambiente con varios obst\'aculos o adversarios movi\'endose
impredeciblemente. Se asume la disponibilidad de informaci\'on perfecta del entorno en todo
momento.

Varias variantes est\'aticas y din\'amicas del algoritmo ``Rapidly Exploring Random
Trees'' (RRT)
se exploran, as\'i como tambi\'en un algoritmo evolutivo para planificaci\'on en ambientes
din\'amicos llamado ``Evolutionary Planner/Navigator.'' Se propone una combinaci\'on de ambos
algoritmos para superar las falencias de ambos y luego una combinaci\'on de RRT
para planificaci\'on inicial y b\'usqueda local informada para navegaci\'on, sumado a una 
heur\'istica voraz simple para optimizaci\'on. Se demuestra que esta combinaci\'on de t\'ecnicas 
simples produce mejores respuestas en ambientes altamente din\'amicos que las variantes RRT
est\'andar.
%Adem\'as se propone la construcci\'on de
%un framework, y su utilizaci\'on para comparar algoritmos que previamente no hab\'ian sido
%comparados.

\textbf{Palabras Claves:} Inteligencia artificial. planificaci\'on de rutas,
RRT, multi-etapa, b\'usqueda local informada, heur\'istica voraz, algoritmos
evolutivos
